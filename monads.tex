\documentclass[a4paper,11pt,twoside, openany]{book}
\usepackage[utf8]{inputenc}
\usepackage{graphicx}
\usepackage{mathrsfs}
\usepackage{amsbsy}
\usepackage{fontenc}
\usepackage{amsfonts}
\usepackage{amsmath} 		
\usepackage{mdframed}
\usepackage{amsthm}  	
\usepackage{amssymb}
\usepackage{amscd}
\usepackage{faktor}
\usepackage{mathtools}
\usepackage{epigraph}
\usepackage{tikz}
\usetikzlibrary{matrix,arrows,decorations.pathmorphing}
\usepackage{tikz-cd}
\usepackage[titletoc]{appendix}
\usepackage{centernot}
\usepackage{bbding}
\usepackage{indentfirst}
\usepackage{hyperref}
\usepackage{xspace}
\hypersetup{colorlinks=false, pdfborder={0 0 0}}                                 
\usepackage[]{enumitem}
\setdescription{font=\normalfont}
%\usepackage[style=alphabetic, backend=bibtex]{biblatex}
\usepackage[normalem]{ulem}
\usepackage{contour}
\makeatletter        

\def\cleardoublepage{\clearpage\if@twoside \ifodd\c@page\else  		
	\hbox{}                                                        					
	\vspace*{\fill}                                                					
	\begin{center}                                                 					
		\*                                                             					
	\end{center}                                                   					
	\vspace{\fill}                                                 					 
	\thispagestyle{empty}                                          				
	\newpage                                                       					
	\if@twocolumn\hbox{}\newpage\fi\fi\fi}                        

\makeatletter
\newcommand{\colim@}[2]{%
	\vtop{\m@th\ialign{##\cr
			\hfil$#1\operator@font colim$\hfil\cr
			\noalign{\nointerlineskip\kern-\ex@}\cr}}%
}
\newcommand{\colim}{%
	\mathop{\mathpalette\colim@{\rightarrowfill@\scriptscriptstyle}}\nmlimits@
}

\newcommand{\catname}[1]{{\normalfont\textbf{#1}}}
\newcommand{\Set}{\catname{Set}}
\newcommand{\sSet}{\catname{sSet}}
\newcommand{\Rel}{\catname{Rel}}
\newcommand{\Cat}{\catname{Cat}}
\newcommand{\CAT}{\catname{CAT}}	
\newcommand{\qCat}{\catname{qCat}}	
\makeatother                                                   					

\makeatletter
\renewcommand\part{%
	\if@openright
	\cleardoublepage
	\else
	\clearpage
	\fi
	\thispagestyle{empty}%  				 
	\if@twocolumn
	\onecolumn
	\@tempswatrue
	\else
	\@tempswafalse
	\fi
	\null\vfil
	\secdef\@part\@spart}
\makeatother
\usepackage{bbm}
\usepackage{fancyhdr}                                   	

\renewcommand{\sectionmark}[1]{\markright{#1}}         	         
\pagestyle{fancy}                                       				
\fancyhf{}                                              				
\fancyhead[LE,RO]{\thepage}                           		           
\fancyhead[LO]{\scshape\nouppercase{\rightmark}}       	          
\fancyhead[RE]{\scshape\nouppercase{\leftmark}}      	           
\renewcommand{\headrulewidth}{0pt}                   \usepackage{amsmath,calligra,mathrsfs}
\DeclareMathOperator{\innerhom}{\mathscr{H}\text{\kern -3pt {\calligra\Large om}}\,}
\DeclareMathOperator{\Hom}{\text{Hom}}
\DeclareMathOperator{\op}{\text{op}}
\DeclareMathOperator{\co}{\text{co}}
\DeclareMathOperator{\coop}{\text{coop}}
\DeclareMathOperator{\Fib}{\text{Fib}}
\DeclareMathOperator{\Cof}{\text{Cof}}
\DeclareMathOperator{\V}{\mathcal{V}}
\DeclareMathOperator{\Ob}{\text{Ob}}
\DeclareMathOperator{\C}{\mathbf{C}}
\DeclareMathOperator{\D}{\mathbf{D}}
\DeclareMathOperator{\N}{\mathbb{N}}
\DeclareMathOperator{\Z}{\mathbb{Z}}
\DeclareMathOperator{\id}{Id}

\tikzset{shorten <>/.style={shorten >=#1,shorten <=#1}}
\newcommand{\pullbackcorner}[1][ul]{\save*!/#1+1.5pc/#1:(1,-1)@^{|-}\restore}
\newcommand{\pushoutcorner}[1][ul]{\save*!/#1-1.5pc/#1:(-1,1)@^{|-}\restore}
\usepackage{tikz}
\usetikzlibrary{shapes}
\usepackage{xcolor}
\usepackage{url}
\usepackage{attachfile} 							
\makeatletter                        						
\g@addto@macro{\UrlBreaks}{\UrlOrds} 				
\makeatother                        						 
\usepackage{xcolor}
\usepackage{url}
\usepackage{attachfile} 							


\makeatletter                        						

\g@addto@macro{\UrlBreaks}{\UrlOrds} 				

\makeatother                        						 

\newsavebox\MBox

\newcommand\Cline[2][red]{{\sbox\MBox{$#2$}%
		
		\rlap{\usebox\MBox}\color{#1}\rule[-1.2\dp\MBox]{\wd\MBox}{1pt}}}

\theoremstyle{definition}

\newtheorem{thm}{Theorem}[section] % reset theorem numbering for each chapter

\newcommand{\chaptercontent}{
	\section{Basics}
	\begin{defn}Here is a new definition.\end{defn}
	\begin{thm}Here is a new theorem.\end{thm}
	\begin{exmp}Here is a good example.\end{exmp}
	\subsection{Some tips}
	\begin{defn}Here is a new definition.\end{defn}
	\section{Advanced stuff}
	\begin{defn}Here is a new definition.\end{defn}
	\subsection{Warnings}
	\begin{defn}Here is a new definition.\end{defn}
}
\theoremstyle{definition}
\newtheorem{defn}[thm]{Definition} % definition numbers are dependent on theor$
\newtheorem{exmp}[thm]{Example} % same for example numbers
\newtheorem{prop}[thm]{Proposition}
\newtheorem{lemma}[thm]{Lemma}
\newtheorem{cor}[thm]{Corollary}
\theoremstyle{remark}
\newtheorem{rmk}[thm]{Remark}
\newtheorem{es}[thm]{Example}

\newmdtheoremenv{theo}[thm]{Theorem}

\newmdtheoremenv{teo}{Theorem}

\setlength{\headheight}{15pt} 					  
\usepackage[all]{xy}
\usepackage{extarrows}


\begin{document}
	
	\author{by \\
		Nicola Di Vittorio \\ Matteo Durante}
	\title{\huge Monads and their applications \\
		\vspace*{5mm}
		\large Dr.\ Daniel Schäppi's course lecture notes} 
	\date{}
	
	\frontmatter
	\maketitle
	\tableofcontents
	
	\chapter{Introduction}
	
	\mainmatter
	
	\chapter{Categorical preliminaries}
	\begin{defn}[Categories]
		
	\end{defn}
	
	\begin{defn}[Functors]
		
	\end{defn}
	
	\begin{defn}[Full functors, faithful functor]
		
	\end{defn}
	
	\begin{defn}[Natural transformations]
		
	\end{defn}
	
	\begin{defn}[Representable Functors]
		
	\end{defn}
	
	\begin{defn}[Whiskering]
		
	\end{defn}
	
	\begin{defn}[Horizontal and vertical composition of nat.transf.]
		
	\end{defn}
	
	\begin{defn}[adjunctions]
		
	\end{defn}
	
	\begin{lemma}[Yoneda]
		
	\end{lemma}
	\begin{proof}
		
	\end{proof}
	
	\chapter{Monads and algebras}
	
	Throughout mathematics we encounter structures defined by some action morphisms. Here we give some examples.
	
	\begin{exmp}
		Given a group $G$, we may consider a $G$-set $X$ described by an action map $G\times X\rightarrow X$.
	\end{exmp}
	\begin{exmp}
		Given an abelian group $M$ and a ring $R$, we can get an $R$-module $M$ by fixing a group homomorphism $R\otimes_{\Z} M\rightarrow M$.
	\end{exmp}
	\begin{exmp}
		Given a monoid $M$ in $Set$, we get a map $\Pi_{k=1}^n M\rightarrow M$, $(m_1,\ldots,m_n)\mapsto ((\ldots ((m_1m_2)m_3)\ldots )m_{n-1}) m_n$. This induces an action map from $W(M)=\amalg_{n\in\N}\Pi_{k=1}^n M$, the set of words on $M$, to $M$.
	\end{exmp}
	\begin{exmp}
		Given a set $X$, let $\mathcal{U}X$ be the set of ultrafilters on it. Any compact T2 topology on $X$ allows us to see each ultrafilter as a system of neighborhoods of a unique point in $X$, hence it gives us a unique map $\mathcal{U}X\rightarrow X$ sending each ultrafilter to the respective point.
	\end{exmp}
	\begin{exmp}
		Given a directed graph $D=(V,E,E\xrightarrow{s}V,E\xrightarrow{t}E)$, we can create its free category $FD$, where the objects are the vertices and $FD(v,w)=\{\text{finite paths } v\rightarrow\ldots\rightarrow w\}$. We set $\id_v$ to be the path of length 0, while composition is just the concatenation of paths.
	\end{exmp}
	\section{Monads}
	\section{Algebras}
	\section{Monadic functors}
	\chapter{Beck’s monadicity theorem}
	\chapter{Monads in 2-category theory}
	\chapter{Monads in $\infty$-category theory}
	
	
	
	
	\backmatter
	% bibliography, glossary and index would go here.
	
\end{document}
