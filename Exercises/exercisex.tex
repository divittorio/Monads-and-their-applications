\documentclass[12pt, a4paper]{article}

\pdfoutput=1

\usepackage{exsheets}

\usepackage{mathrsfs, amsmath, amsthm, amssymb, stmaryrd, enumerate}%
\usepackage{tikz}%
\usepackage[linktocpage]{hyperref}%
%\usepackage{cleveref}
\usepackage[all]{xy}%
\xyoption{2cell}%
\UseAllTwocells%

\newdir{t>}{{\UseTips\dir{>}}}

%\creflabelformat{enumi}{(#2#1#3)}

%\hypersetup{
% pdfauthor={Daniel Schaeppi},
% pdfkeywords={Pushouts} {algebraic stacks} {Quasi-coherent sheaves} {weakly Tannakian categories}
%}


%********************************* MACROS ************************************%

\DeclareMathOperator{\Ob}{Ob}
\DeclareMathOperator{\id}{id}
\DeclareMathOperator{\el}{el}
\DeclareMathOperator{\op}{op}
\DeclareMathOperator{\Lan}{Lan}
\DeclareMathOperator{\Ran}{Ran}
\DeclareMathOperator{\Mod}{\mathbf{Mod}}
\DeclareMathOperator{\Rep}{Rep}
\DeclareMathOperator{\Vect}{\mathbf{Vect}}
\DeclareMathOperator{\Coalg}{\mathbf{Coalg}}
\DeclareMathOperator{\fgp}{fgp}
\DeclareMathOperator{\fp}{fp}
\DeclareMathOperator{\fd}{fd}
\DeclareMathOperator{\Hom}{Hom}
\DeclareMathOperator{\End}{End}
\DeclareMathOperator{\Spec}{Spec}

\DeclareMathOperator{\Tor}{Tor}
\DeclareMathOperator{\reg}{reg}
\DeclareMathOperator{\flt}{flat}


\DeclareMathOperator{\rk}{rk}
\DeclareMathOperator{\Sym}{Sym}
\DeclareMathOperator{\sgn}{sgn}
\DeclareMathOperator{\coev}{coev}
\DeclareMathOperator{\ev}{ev}

\DeclareMathOperator{\fl}{fin. \ell}

\DeclareMathOperator{\fin}{fin}
\DeclareMathOperator{\fg}{fg}
\DeclareMathOperator{\idem}{idem}
\DeclareMathOperator{\Modtwo}{\mathcal{M}}

\DeclareMathOperator{\iso}{iso}

\DeclareMathOperator{\LFdi}{\LF^{\rk d}_{\iso}}

\DeclareMathOperator{\dom}{dom}
\DeclareMathOperator{\Sh}{Sh}
\DeclareMathOperator{\Rex}{\mathbf{Rex}}

\DeclareMathOperator{\SymMonCat}{SymMonCat}


\DeclareMathOperator{\PsCone}{PsCone}


\DeclareMathOperator{\Ind}{Ind}
\DeclareMathOperator{\Span}{\mathbf{Span}}
\DeclareMathOperator{\Cospan}{\mathbf{Cospan}}
\DeclareMathOperator{\EM}{EM}

\DeclareMathOperator{\Aff}{\mathbf{Aff}}
\DeclareMathOperator{\Alg}{\mathbf{Alg}}
\DeclareMathOperator{\Coh}{\mathbf{Coh}}
\DeclareMathOperator{\QCoh}{\mathbf{QCoh}}
\DeclareMathOperator{\VB}{\mathbf{VB}}

\DeclareMathOperator{\BGL}{\mathrm{BGL}}

\DeclareMathOperator{\Cov}{Cov}
\DeclareMathOperator{\fpqc}{\mathit{fpqc}}

\DeclareMathOperator{\Aut}{Aut}
\DeclareMathOperator{\lax}{lax}
\DeclareMathOperator{\Tors}{Tors}


\DeclareMathOperator{\Map}{Map}
\DeclareMathOperator{\target}{target}
\DeclareMathOperator{\cospan}{Cospan}
\DeclareMathOperator{\CommAlg}{\mathbf{CommAlg}}
\DeclareMathOperator{\coop}{coop}
\DeclareMathOperator{\Coact}{Coact}
\DeclareMathOperator{\Gray}{\mathbf{Gray}}
\DeclareMathOperator{\PsMon}{\mathbf{PsMon}}
\DeclareMathOperator{\BrPsMon}{\mathbf{BrPsMon}}
\DeclareMathOperator{\SymPsMon}{\mathbf{SymPsMon}}
\DeclareMathOperator{\MonComon}{\mathbf{MonComon}}
\DeclareMathOperator{\BrMonComon}{\mathbf{BrMonComon}}
\DeclareMathOperator{\SymMonComon}{\mathbf{SymMonComon}}
\DeclareMathOperator{\Psa}{\mathbf{Psa}}

\DeclareMathOperator{\pr}{pr}

\DeclareMathOperator{\len}{\ell}
\DeclareMathOperator{\proj}{proj}
\DeclareMathOperator{\Fil}{Fil}
\DeclareMathOperator{\MF}{MF}
\DeclareMathOperator{\colim}{colim}
\DeclareMathOperator{\Er}{Er}

\DeclareMathOperator{\Ps}{\mathbf{Ps}}

\DeclareMathOperator{\Cocts}{\mathbf{Cocts}}
\DeclareMathOperator{\Lex}{\mathbf{Lex}}
\DeclareMathOperator{\cat}{\mathbf{cat}}
\DeclareMathOperator{\Cat}{\mathbf{Cat}}
\DeclareMathOperator{\Bicat}{\mathbf{Bicat}}
\DeclareMathOperator{\Tricat}{\mathbf{Tricat}}
\DeclareMathOperator{\Gpd}{\mathbf{Gpd}}
\DeclareMathOperator{\Mky}{\mathbf{Mky}}
\DeclareMathOperator{\CAT}{\mathbf{CAT}}
\DeclareMathOperator{\Set}{\mathbf{Set}}
\DeclareMathOperator{\Ab}{\mathbf{Ab}}
\DeclareMathOperator{\CGTop}{\mathbf{CGTop}}
\DeclareMathOperator{\Mon}{\mathbf{Mon}}
\DeclareMathOperator{\Comon}{\mathbf{Comon}}
\DeclareMathOperator{\Comod}{\mathbf{Comod}}

\DeclareMathOperator{\Nat}{\mathrm{Nat}}
\DeclareMathOperator{\Fun}{\mathrm{Fun}}
\DeclareMathOperator{\LF}{\mathrm{LF}}

\DeclareMathOperator{\CAlg}{\mathrm{CAlg}}

\newcommand{\ca}[1]{\mathscr{#1}}
\newcommand{\VNat}{\ca{V}\mbox{-}\Nat}
\newcommand{\Vcat}{\ca{V}\mbox{-}\Cat}
\newcommand{\VCAT}{\ca{V}\mbox{-}\CAT}
\newcommand{\VCATlp}{\ca{V}\mbox{-}\CAT_{\mathrm{lp}}}
\newcommand{\VCATc}{\ca{V}^{\prime}\mbox{-}\Cat_{\mathrm{c}}}

\newcommand{\Catfc}{\Cat_{\mathrm{fc}}}
\newcommand{\VCatfc}{\ca{V}\mbox{-}\Cat_{\mathrm{fc}}}
\newcommand{\kten}{\mathop{\boxtimes_{\mathrm{fc}}}}

\newcommand{\Prs}[1]{\mathcal{P}\ca{#1}}
\newcommand{\Bimod}[1]{{_\ca{#1}}{\mathcal{M}}{_\ca{#1}}}
\newcommand{\modules}[2]{{_\ca{#1}}{\mathcal{M}}{_\ca{#2}}}
\newcommand{\CC}[1]{\mathbf{Comon}\left(\Bimod{#1}\right)}

\DeclareMathOperator{\U}{O}

\newcommand{\ubar}[1]{\underline{#1\mkern-4mu}\mkern4mu }

\newcommand{\dual}[1]{{#1}^{\circ}}
\newcommand{\ldual}[1]{{#1}^{\vee}}


\newcommand{\ten}[1]{\mathop{{\otimes}_{#1}}}
\newcommand{\tenl}[1]{\mathop{{}_{#1}{\otimes}}}
\newcommand{\tenlr}[2]{\mathop{{}_{#1}{\otimes}_{#2}}}

\newcommand{\boxten}[1]{\mathop{{\boxtimes}_{#1}}}

\newcommand{\pb}[1]{\mathop{{\times}_{#1}}}
\newcommand{\po}[1]{\mathop{{+}_{#1}}}


\newcommand{\defl}{\mathrel{\mathop:}=}


% THEOREM ENVIRONMENTS

\theoremstyle{plain}
\newtheorem{thm}{Theorem}[subsection]
\newtheorem*{thm*}{Theorem}
\newtheorem{prop}[thm]{Proposition}
\newtheorem{lemma}[thm]{Lemma}
\newtheorem{cor}[thm]{Corollary}

\theoremstyle{definition}
\newtheorem{example}[thm]{Example}
\newtheorem{rmk}[thm]{Remark}
\newtheorem{dfn}[thm]{Definition}
\newtheorem{notation}[thm]{Notation}

\newtheoremstyle{citing}{}{}{\itshape}{}{\bfseries}{.}{ }{\thmnote{#3}}
\theoremstyle{citing}
\newtheorem{cit}{}

\newtheoremstyle{citingdfn}{}{}{}{}{\bfseries}{.}{ }{\thmnote{#3}}
\theoremstyle{citingdfn}
\newtheorem{citdfn}{}


\numberwithin{equation}{section}

%\keywords{Fiber functors, Tannakian categories}
%\subjclass[2010]{14A20, 18D10}

%\author{Daniel Sch\"appi}
%\thanks{This research was supported by the DFG grant: SFB 1085 ``Higher invariants.''}
%\address{Fakult{\"a}t f{\"u}r Mathematik,
%Universit{\"a}t Regensburg,
%93040 Regensburg,
%Germany}
%\email{daniel.schaeppi@ur.de}
\date{}


\title{Monads and their applications -- Sheet 1}




\begin{document}

\SetupExSheets{
 headings=block-subtitle,
}

\pagestyle{empty}
%\maketitle
\section*{Monads and their applications 10}

\begin{question} 
 Let $\ca{V}$ be a braided monoidal category. Show that the monoidal structure of $\ca{V}$ lifts to a monoidal structure of $\Mon(\ca{V})$ (that is, put a monoid structure on $M \otimes M^{\prime}$ so that $\alpha$, $\rho$, $\lambda$ become monoid morphisms). If $\ca{V}$ is symmetric, show that $\gamma$ lifts to a symmetry on $\Mon(\ca{V})$.
 \end{question}

\begin{question}
  Show that a monoid in $\Mon(\ca{V})$ is a commutative monoid (this is known as the ``Eckmann-Hilton argument''). More precisely, if $\bigl((M,\mu,\eta),\mu^{\prime},\eta^{\prime}\bigr)$ is a monoid in $\Mon(\ca{V})$, show that $\eta^{\prime}=\eta$, $\mu^{\prime}=\mu$ and $\mu \gamma_{M,M}=\mu$. (Hint: it is best to first prove this for $\ca{V}=\Set$ and then translate to general monoidal categories. First show that the two units $\eta=\eta^{\prime}$ are the same and then turn the following pictorial ``proof'' into a precise argument, where we write one multiplication vertically, one horizontally:
  \[
   \begin{bmatrix}
 a & b \\
  \end{bmatrix} \;=\;
   \begin{bmatrix}
 a & \eta  \\
 \eta & b
  \end{bmatrix} \;=\;
   \begin{bmatrix}
 \eta & a \\
 b & \eta
  \end{bmatrix} \;=\;
   \begin{bmatrix}
 b & a\\
  \end{bmatrix} 
  \]
 for all $a,b \in M$.
\end{question}

\begin{question}
 Let $\ca{V}$ be a symmetric monoidal closed category, $\ca{C}$ a $\ca{V}$-category, and $C \in \ca{C}$. Show that the functor $\ca{C}(C,-) \colon \ca{C} \rightarrow \ca{V}$ defined in class is indeed a $\ca{V}$-functor.
\end{question}


\begin{question}
 Let $\ca{V}$ be a symmetric monoidal category. If $A \in \ca{V}$ is a monoid, then $A\otimes-$ is a monad. The $(A\otimes-)$-aglebras are called \emph{$A$-modules} and we write $\ca{V}_A$ for $(A\otimes-)\mbox{-}\Alg$. Now suppose that $A$ is commutative, that $\ca{V}$ has reflexive coequalizers, and that $V\otimes-$ preserves these for all $V \in \ca{V}$. Given $A$-modules $M$ and $N$, define $M \ten{A} N$ by the reflexive coequalizer
 \[
 \xymatrix{M \otimes A \otimes N \ar@<1ex>[r] \ar@<-1ex>[r] & M \otimes N \ar[l] \ar[r] & M\ten{A} N} 
 \]
 in $\ca{V}$. Show that this lifts to a functor $-\ten{A}- \colon \ca{V}_A \otimes \ca{V}_A \rightarrow \ca{V}_A$ making $\ca{V}_A$ into a symmetric monoidal category with unit $A$ (Hint: use the fact that reflexive coequalizers are sifted and split coequalizers are absolute).
 \end{question}

\begin{question}[subtitle=(bonus)]
 Give an example of a braided monoidal category $\ca{V}$ such that the braiding does not lift to $\Mon(\ca{V})$.
\end{question}



\end{document}
